\documentclass{article}
\usepackage{amsmath,amsfonts,amssymb,fullpage}

\newcommand\ZZ{\mathbb Z}
\newcommand\Zm[1]{\ZZ/#1\ZZ}
\DeclareMathOperator\Aut{Aut}
\DeclareMathOperator\Hom{Hom}
\DeclareMathOperator\im{im}

\title{Algebra Homework 1}
\author{Mendel Feygelson}

\begin{document}
\maketitle
\begin{enumerate}

   \item
      \begin{enumerate}

         \item We take $N$ to be the subgroup consisting of upper triangular
            matrices with 1s on the diagonal. We note that when multiplying two
            matrices in $A \in B$, their diagonal entries multiply termwise, so
            a diagonal entry of $A^{-1}$ is the inverse of the corresponding
            diagonal entry of $A$, and so if $X \in N$, we see that the diagonal
            entries of $A^{-1}XA$ are 1, so $N$ is normal. Clearly $N \cap T =
            1$, and if we have a matrix $(a_{ij}) \in B$, then multiplying the
            matrix in $T$ whose diagonal is given by $(a_{ii})$ by the matrix in
            $N$ whose entries are given by $(\frac{a_{ij}}{a_{ii}})$ (on the
            right) gives us back $A$, so $B = TN$.

         \item $T$ is abelian, so we need to show that $N$ is solvable. We
            consider the subspace of $R^n$ where the last entry is zero. $N$
            acts on this space like the corresponding group in dimension $n-1$.
            And the kernel of this action, the subgroup where all of the
            non-diagnal entries outisde the last column are zero, is isomorphic
            to the additive group of $n-1$, so is abelian. So by induction, we
            have that $N$ is solvable.

      \end{enumerate}

   \item 
      \begin{enumerate}

         \item The commutator of $S_4$ contains every element of the form
            $\sigma\sigma^\tau$, with $\sigma$ and $\tau$ in $S_4$. In
            particular, since the 2-cycles form a single conjugacy class, it
            contains every element that can be written as the product of two
            2-cycles, and these generate $A_4$, so $S_4' = A_4$. But $A_4$ is
            not abelian, so $S_4$ is not metabelian. However, $A_4$ is solvable
            as we'll see later, and $S_4/A_4$ has order 2, so $S_4$ is solvable.
         
         \item We've seen that every group of order $p^2$ is abelian, so every
            group of order $p^4$ is metabelian (we use the fact that a $p$-group
            has a normal subgroup corresponding to every power of $p$ dividing
            the order of the group, so in particular a group of order $p^4$ has
            a normal subgroup of order $p^2$). So all $p$-groups of order less
            than 24 are metabelian. We've also seen that groups of order $pq$
            are metabelian. Thus the only orders that remain to check are 12,
            18, and 20. But these all have order $p^2q$ and all groups of order
            $p^2q$ are metabelian as a consequence of the solution to 3(a).

      \end{enumerate}

   \item
      \begin{enumerate}

         \item If $p^2 < q$, then the Sylow $q$-subgroup is normal and solvable,
            and we know that groups of order $p^2$ are solvable, so $G$ is
            solvable.

            Suppose $p^2 > q$, and suppose the Sylow $p$-subgroup is not normal.
            That means that there are $q$ of them, and that $q \equiv 1 \mod p$.
            Now if the Sylow $q$-subgroup is not normal, then there must be at
            least $p$ of them. If there are exactly $p$, then $p \equiv 1 \mod
            q$, which can't happen since $q > p$. So there must be $p^2$
            $q$-subgroups. Distinct $q$-groups can only intersect at the
            identity, so that's $p^2q - p^2$ elements of order $q$. But that
            leaves only $p^2$ elements not of order $q$, so there can only be
            one Sylow $p$-subgroup, which is normal. So either the Sylow
            $p$-subgroup or Sylow $q$-subgroup is normal, so by taking the
            quotient we see that $G$ is solvable.

         \item The only orders remaining to check are $24 = 2^3\cdot3$, $30 =
            2\cdot3\cdot5$, $32 = 2^5$, $36 = 2^2\cdot3^2$, $40=2^3\cdot5$, $42
            = 2\cdot3\cdot7$, $48 = 2^4\cdot3$, $54=2\cdot3^3$, and
            $56=2^3\cdot7$. $p$-groups are always solvable, so groups of order
            32 aren't a concern, and a group of order 54 has a normal Sylow
            subgroup of order 27, so is solvable.

            In the case of 30, there can be either one or six Sylow 5-subgroups.
            If there are six, that accounts for 24 elements of order 5, so there
            cannot be ten Sylow 3-subgroups.

            In the case of 40, there can only be one Sylow 5-subgroup. In the
            case of 42, there can only be one Sylow 7-subgroup. In the case of
            56, there must be eight Sylow 7-subgroups, yielding 48 elements of
            order 7, and only enough room for one Sylow 2-subgroup.

            That leaves 24, 36, and 48. I'm going to work out the technique to
            do these some other time.

      \end{enumerate}

   \item There is of course the direct product $\Zm{p^2} \times \Zm{p}$. To get
      a different semidirect product, we need a nontrivial homomorphism $\Zm p
      \to \Aut(\Zm{p^2}) = (\Zm{p^2})^\times$. $(\Zm{p^2})^\times$ is a cyclic
      group of order $p(p-1)$, so is isomorphic to $\Zm{p} \times \Zm{p-1}$. So
      we can get a semidirect product by mapping $\Zm{p}$ to the subgroup of
      $\Aut(\Zm{p^2})$ isomorphic to $\Zm{p}$.

   \item First, if $q=2$, then $b=b^{-1}$, and this is just the usual
      presentation of the dihedral group $D_{2p}$, and similarly if $p=2$, this
      is just the usual presentation of $D_{2q}$ (the relation $ab^{-1}a = b$
      follows from the relation $aba = b^{-1}$). Suppose $p$ and $q$ are both
      odd.  We have $a^{-1} = bab = b^{-1}ab^{-1}$, so $a^{-2} = b^{-1}a^2b$,
      and if $r$ is the inverse of 2 modulo $p$, then taking this relation to
      the power $r$, we see that $a^b = a^{-1}$. But then $1 = abab = b^2 =
      b^q$, so $b=1$, so $1 = a^2 = a^p$, so $a=1$ and the group is trivial.

   \item $f_*$ takes a map $h : G \to A$ to $f \circ h : G \to B$, and $f^*$
      takes a map $h : B \to G$ to $h \circ f : A \to G$. These are
      homomorphisms since $f$ is a homomorphism. $f_*$ is injective since $f$ is
      injective ($f$ has trivial kernel, so if $f_*(h)$ is trivial, then $\im h
      \subset \ker f = 1$, so $g$ is trivial). $\im(g_*(f_*(h)) \subset \im(g
      \circ f) = 1$, so $g_* \circ f_* = 1$. And if $g_*(h) = 1$ then $\im h
      \subset \ker g = \im f$, and so since $f$ injects $A$ into $B$, it induces
      an isomorphism $A \cong \im f$, so this gives us a map $\tilde h : G \to
      \im f \cong A$ such that $f \circ \tilde h = h$. So the first diagram is
      exact.

      Similarly, $g^*$ is injective because $g$ is surjective -- if $g^*(h) =
      1$, then $h$ must be identically 1 on $C$. And $f^*(g^*(h)) = h \circ g
      \circ f \equiv 1$ since $g \circ f = 1$. And if we have $h : B \to G$ such
      that $h \circ f = 1$, so $h$ is trivial on the image of $f$, which is the
      kernel of $g$, then $h$ factors through $g$, so $h$ = $\tilde h \circ g$
      for $\tilde h : C \to G$, so $h$ is in the image of $g^*$.

      Finally, consider the exact sequence $1 \to \ZZ \to \ZZ \to \Zm2 \to 1$,
      where the first map is multiplication by 2. We have the identity map $\Zm2
      \to \Zm2$, but that doesn't factor through $\ZZ$ since every map $\Zm2 \to
      \ZZ$ is trivial, so the first sequence need not be surjective. Also, the
      identity map $\ZZ \to \ZZ$ doesn't factor through multiplication by 2, so
      we see that the second sequence also isn't surjective.

   \item An element of $\Hom(G,H_1) \times_{\Hom(G,K)} \Hom(G,H_2)$ is given by
      a pair of maps $G \to H_1$ and $G \to H_2$ that give the same map to $K$.
      But then by the universal property of pullback, this gives us a unique map
      $G \to H_1 \times_K H_2$, and conversely, any map $G \to H_1 \times_K H_2$
      gives us maps $G \to H_1$ and $G \to H_2$ which give the same map to $K$.
      And it's pretty clear that this identification is natural.

      Similarly, an element of $\Hom(H_1,G) \times_{\Hom(K,G)} \Hom(H_2,G)$
      is given by two maps $H_1 \to G$ and $H_2 \to G$ which agree on $K$, and
      this is precisely what maps $H_1 *_K H_2 \to G$ are.

   \item It suffices to check that the group defined by these generators and
      relations satisfies the universal property of coproduct. Call this group
      $E$. We have a map $S_1 \to E$, which lifts to a map $F(S_1) \to E$, and
      since this map is zero on $R_1$, it factors through $G$, so we have a map
      $G \to E$, and similarly we have a map $H \to E$. Now suppose we have maps
      $f$ and $g$ from $G$ and $H$ respectively to a group $X$. These give us
      maps $S_1 \to G \stackrel{f}\to X$ and $S_2 \to H \stackrel{g}\to X$, and
      so we have maps from the free groups. And these maps are zero on $R_1$ and
      $R_2$, so this gives us a map $E \to X$, which is uniquely determined
      since $f$ and $g$ determine where the generators go.

\end{enumerate}
\end{document}
