\documentclass{article}
\usepackage{amsmath,amsfonts,amssymb}
\usepackage{fullpage}
%\usepackage{mathrsfs}

\newcommand\ZZ{\mathbb Z}
\newcommand\RR{\mathbb R}
\newcommand\CC{\mathbb C}
\newcommand\QQ{\mathbb Q}
\newcommand\C{\mathcal C}
\newcommand\D{\mathcal D}
\newcommand\F{\mathcal F}
\newcommand\G{\mathcal G}
\DeclareMathOperator\Hom{Hom}
\DeclareMathOperator\Gal{Gal}
\DeclareMathOperator\Aut{Aut}
\DeclareMathOperator\sgn{sgn}

\title{Algebra Homework 2}
\author{Mendel Feygelson}

\begin{document}
\maketitle
\begin{enumerate}

   \item Given a diagram $\{X_i \to X_j : i < j\}$ with a limit $\varinjlim
      X_i$, since $F$ is a functor, we have maps $F(X_i) \to F(\varinjlim X_i)$
      compatible with the maps $F(X_i) \to F(X_j)$. To show that this is the
      direct limit, we need to show that given a $Y \in \D$, with compatible
      maps $F(X_i) \to Y$, we have a unique map $F(\varinjlim X_i) \to Y$
      compatible with all of the other maps. But the maps $F(X_i) \to Y$
      correspond to maps $X_i \to G(Y)$, which give us a unique compatible map
      $\varinjlim X_i \to G(Y)$, and so there is in fact a unique compatible map
      $F(\varinjlim X_i) \to Y$, since $\Hom_\D(F(\varinjlim X_i), Y) \cong
      \Hom_\C(\varinjlim X_i, G(Y))$.

   \item Clearly $k(\alpha^2) \subset k(\alpha)$, so we just need to check that
      $\alpha \in k(\alpha^2)$. Say $\alpha^d = \sum_{i=0}^{d-1} a_i\alpha^i$,
      for $d$ odd. Then we write $\alpha^{d+1} = \alpha\sum_{i=0}^{\frac{d-1}2}
      a_{2_i}\alpha^{2i} + \sum_{i=1}^{\frac{d-1}2} a_{2i-1}\alpha^{2i}$, so we
      see that $\alpha \in k(\alpha^2)$.

   \item Let $\alpha$ be a root of $X^p-a$. If the characteristic of $k$ is $p$,
      then $X^p-a = (X-\alpha)^p$. And any irreducible polynomial of
      characteristic less than $p$ must have distinct roots, so we see that a
      monic irreducible factor of $X^p-a$ must be either $X^p-a$ or $X-\alpha$.

      Now if the characteristic of $k$ is not $p$, let $\zeta$ denote a
      primitive $p$th root of unity. Then the roots of $X^p-a$ are given by
      $\zeta^i\alpha$ for $0 \leq i < p$. So
      \[ X^p - a = \prod_{i=0}^{p-1} (X-\zeta^i\alpha) \]
      So any factor of $X^p-a$ will have a constant term that looks like
      $\zeta^i\alpha^n$. Suppose $\zeta^i\alpha^n \in k$ for some $0<n<p$. Let
      $m,d \in \ZZ$ such that $mn = dp + 1$. Then $(\zeta^i\alpha^n)^m =
      \zeta^{im}\alpha\cdot a^d \in k$, so $\zeta^{im}\alpha \in k$. So then $k$
      contains a root of $X^p-a$. So if $k$ doesn't contain any roots of $X^p-a$
      then $k[X]$ doesn't contain any proper divisors of $X^p-a$.

      Finally, note that $X^4+1 = (X^2 + \sqrt2X + 1)(X^2 - \sqrt2X + 1)$ has no
      roots in $\RR$, so we need $p$ prime.

   \item Let $\alpha \in K$. We let $f_\alpha = \prod X-\beta$, where $\beta$
      ranges over the $G$-orbit of $\alpha$. This polynomial is fixed by $G$, so
      is in $K^G[X]$. It has $\alpha$ as a root, it has distinct roots, and all
      of its roots are in $K$. So $K / K^G$ is normal and separable. And I'm
      hoping that the proof from class $[K:K^G] \leq |G|$ follows through to
      show that it is finite. Finally, it's clear that $G \leq \Gal(K/K^G)$, and
      then we have that $|\Gal(K/K^G)| \leq [K:K^G] \leq |G|$, so $G =
      \Gal(K/K^G)$.

   \item We know that if $\alpha$ is separable and $\beta$ is separable, then
      $k[\alpha,\beta]$ is separable (it's contained in the splitting field of
      the product of the minimal polynomials of $\alpha$ and $\beta$, which are
      separable), so clearly the set of separable elements forms a subfield, and
      it's clear that it's the maximal separable intermediate field.

      Finally, if $\alpha \in K$ is separable over $K_s$, then $K_s[\alpha]$ is
      separable over $k$. (Say, taking normal closures, we have $k \subset
      K_{sn} \subset L$, where $L$ is the splitting field of the minimal
      polynomial of $\alpha$ over $K_{sn}$. Then these are both Galois
      extensions, so nothing in $L \setminus K_{sn}$ is fixed by
      $\Gal(L/K_{sn})$ and nothing in $K_{sn}/k$ is fixed by $\Gal(K_{sn}/k)$,
      and since $L$ is a splitting field, every $\sigma \in \Gal(K_{sn}/k)$
      extends to a $\overline\sigma \in \Gal(L/k)$. So nothing in $L \setminus
      k$ is fixed by $\Gal(L/k)$, so $L/k$ is Galois, so $\alpha \in L$ is
      separable over $k$.) Thus, any $\alpha \in K \setminus K_s$ is inseparable
      over $K_s$. Let $f$ be its minimal polynomial over $K_s$. We claim that
      $f$ is of degree $p^n$. We know that $f = g(X^p)$ for some $g \in K_s[X]$.
      Suppose $g$ is separable. Then $\alpha^p$ is a root of $g$, so is in
      $K_s$. So $\alpha$ is a root of $X^p - \alpha^p$, so $f | X^p - \alpha^p$,
      so $\deg f = p$. Otherwise, $g$ is inseparable, so by induction on degree,
      $\deg g = p^{n-1}$ for some $n$, so $\deg f = p^n$. Finally, I need to
      think about what happens when I adjoin multiple inseparable elements, but
      I don't know how to do that.

   \item If $\alpha \in K_s$, then its minimal polynomial is separable so all of
      its roots are separable, so all of its roots are in $K_s$ so $K_s/k$ is
      Galois. And $\sigma \in \Gal(K_s/k)$ lifts to something in $\Aut_k(K)$
      since $K$ is a splitting field, and this lift must be  unique since, from
      the argument in the last section, we see that every $\alpha \in K$ has
      minimal polynomial $(X-\alpha)^{p^n}$ over $K_s$, and so must be fixed by
      any $\tau \in \Aut_{K_s}(K)$. So $\Gal(K_s/k) \cong \Aut_k(K)$.

   \item We note that a permutation $\sigma$ of the $a_i$ takes $\Delta$ to
      $\sgn(\sigma)\Delta$. So precisely the even permutation in $G$ fix
      $\Delta$.

   \item $K = \QQ[\sqrt2,\sqrt[3]3,\zeta]$, where $\zeta$ is a primitive cube
      root of unity. The Galois group $G$ may permute $\{\pm\sqrt2\}$
      $\{\zeta^i\sqrt[3]3\}$, and $\{\zeta,\zeta^2\}$. We let
      $\sigma:\sqrt2\mapsto-\sqrt2$, $\tau:\sqrt[3]3\mapsto\zeta\sqrt[3]3$, and
      $\psi:\zeta\mapsto\zeta^2$. Then $\sigma$ is central and $\tau\psi =
      \psi\tau^{-1}$. So $G = \langle \sigma \rangle_2 \times (\langle \tau
      \rangle_3 \rtimes \langle \psi \rangle_2) \cong C_2 \times S_3$. The
      Galois subfields correspond to normal subgroups, so are $\QQ$,
      $\QQ[\zeta]$, $\QQ[\zeta, \sqrt[3]3]$, and each of those with $\sqrt2$
      adjoined.

\end{enumerate}
\end{document}
