\documentclass{article}
\usepackage{amsmath,amsfonts,amssymb}
\usepackage{fullpage}
%\usepackage{mathrsfs}

\newcommand\ZZ{\mathbb Z}
\newcommand\RR{\mathbb R}
\newcommand\CC{\mathbb C}
\newcommand\QQ{\mathbb Q}
\newcommand\NN{\mathbb N}
\newcommand\FF{\mathbb F}

\renewcommand\a[1]{\left\langle#1\right\rangle}
\renewcommand\b[1]{\left\{#1\right\}}

\DeclareMathOperator\Hom{Hom}
\DeclareMathOperator\Aut{Aut}
\DeclareMathOperator\Gal{Gal}
\DeclareMathOperator\ab{ab}

\title{Algebra Homework 3}
\author{Mendel Feygelson}

\begin{document}
\maketitle
\begin{enumerate}

   \item Let $K = \QQ[\sqrt2,\sqrt{3+\sqrt3}]$. Let $\beta = \sqrt{3+\sqrt3}$.
      The roots of the minimal polynomial of $\sqrt2$, which are $\pm\sqrt2$ are
      in $K$. $\beta$ satisfies $(x^2-3)^2=3$, and the other roots of that
      polynomial are $\pm\sqrt{3\pm\sqrt3}$. So to see that $K$ is normal, and
      therefore Galois, it suffices to check that $\sqrt{3-\sqrt3} \in K$. It's
      easy to see that $\sqrt3 \in K$, and so $\sqrt6 \in K$, and this element
      is just $\sqrt6 - \beta \in K$. So $K/\QQ$ is Galois.

      Let $\sigma$ swap $\{\pm\sqrt2\}$, $\tau$ swap $\{\pm\sqrt{3+\sqrt3}\}$,
      and $\omega$ swap $\{\sqrt{3\pm\sqrt3}$ (and also their negatives).
      Writing $\beta' = \sqrt{3-\sqrt3}$, we see that $\langle\omega,\tau\rangle
      = D_8$, where $\omega\tau = (\beta'\,\beta\,-\beta'\,-\beta)$ is a
      rotation and $\omega=(\beta'\,\beta)(-\beta'\,-\beta)$ is a reflection.
      And $\sigma$ is central so the Galois group is $C_2 \times D_8$.
      
      Writing $r = \omega\tau$ and $t = \omega$, we find the subgroups of $D_8$:
      $\a1$, $\a{r}$, $\a{r^2}$, $\a{t}$, $\a{rt}$, $\a{r^2t}$, $\a{r^3t}$,
      $\a{r^2,t}$, and $\a{r,t}$.
      
      Then, on second thought $\sigma$ may not be central because
      $\sqrt{3+\sqrt3}\sqrt{3-\sqrt3}=\sqrt2\sqrt3$, but I'm out of time.

   \item Shamelessly copying from Wikipedia, if we let $k = \FF_p(T,S)$, where
      $T$ and $S$ are transcendental, and $K = k(t,s)$, where $t^p = T$ and
      $s^p = S$, then $[K:k] = [K:k(s)][k(s):k] = p^2$. But for every $\alpha
      \in K$, $\alpha^p \in k$, so $[k(\alpha):k] \leq p$, so this extension has
      no primitive element. So by the primitive element theorem, there must be
      infinitely many intermediate fields.

   \item
      \begin{enumerate}
         \item It's clear how $\Aut_k(K) \hookrightarrow \Hom_k(K,L)$, so we
            need to show that $K/k$ is normal if and only if there's nothing
            else in $\Hom_k(K,L)$, that is, $K$ embeds into every $k$-extension
            uniquely. Well, if there's some other way to embed $K$ into $L$, say
            $K' \subset L$, $K' \cong K$. Then, there's some element $\alpha
            \in K' \setminus K$, and if $f$ is the minimal polynomial of
            $\alpha$, then $K$ contains some root of $f$ since it's isomorphic
            to $K'$ but doesn't contain $\alpha$ so is not normal. Conversely,
            if $K/k$ is not normal, then let $\alpha \in K$ such that its
            minimal polynomial has a root $\beta$ in $L \setminus K$, where $L$
            is some extension. Then $k[\beta] \cong k[\alpha]$ can be extended
            to an different embedding $K \hookrightarrow L$ which contains
            $\beta$. And this embedding is in $\Hom_k(K,L) \setminus \Aut_k(K)$.

         \item It's not clear to me how to canonically extend an element of
            $\Gal(K/k)$ to an element of $\Gal(L/k)$ for $K \subset L$, but if
            you can do this then it's clear that this gives a functor.
      \end{enumerate}

   \item $k_p$ is perfect since if $f$ is an irreducible polynomial in $k_p[X]$
      with repeated roots then it's a polynomial over some finite purely
      inseparable extension of $k$, $K \subset k_p$. So if $\alpha$ is a root of
      $f$ then $K[\alpha]$ is a finite purely inseparable extension of $k$, so
      $\alpha \in k_p$, a contradiction since $f$ is irreducible. Conversely, if
      $k \subset K \subset k_p$ with $K$ perfect, then $k_p$ is separable over
      $K$. We saw last time that any $\alpha \in k_p$ would have minimal
      polynomial $(X-\alpha)^{p^n}$ over $k$, where $p$ is the characteristic.
      So for $\alpha$ to be separable over $K$, we must have that $\alpha \in
      K$, so $k_p = K$.

   \item By factoring polynomials, we see that every nonconstant polynomial in
      $k[X]$ has all of its roots in $K$. But then every element of the
      algebraic closure of $k$ is algebraic over $k$ and so is in $K$. So $K =
      \overline k$ is algebraically closed.

   \item
      \begin{enumerate}
     
         \item $n\hat\ZZ$ is closed, so we have that $\ZZ/n\ZZ \subset
            \hat\ZZ/n\hat\ZZ$ is finite and dense, and $\hat\ZZ/n\hat\ZZ$ is
            Hausdorff, so $\hat\ZZ/n\hat\ZZ = \overline{\ZZ/n\ZZ} = \ZZ/n\ZZ$.

         \item Since they have finite index, the subgroups given by $n\hat\ZZ$
            are both open and closed. And any open subgroup must be given by
            its residues modulo finitely many $n$, so by the Chinese Remainder
            Theorem, these should be the only open subgroups.

      \end{enumerate}

   \item 

      \begin{enumerate}
         \item $k^{\ab}$ is clearly Galois since it's the limit of Galois
            extensions. So then it suffices to show that $\Gal(\bar k_s /
            k^{\ab}) = \overline{G_k'}$, or that $k^{\ab}$ is the fixed field of
            $G_k'$. And this is true because a normal extension is abelian
            precisely when it's fixed by the commutator, and every extension is
            contained in a normal extension.

         \item $\QQ^{\ab}$ is obtained by adjoining all roots of unity, so the
            Galois group is just $\varprojlim (\ZZ/n\ZZ)^\times$.
      \end{enumerate}

   \item 

      \begin{enumerate}
         \item Let $S$ be such a basis. Then $\CC = \overline{\QQ(S)}$.  Assume
            $S$ is infinite for simplicity. Then one can estimate $|\QQ(S)| \leq
            |\QQ \times S \times \NN| = |S|$. And $|\overline{\QQ(S)}| \leq
            |\QQ(S) \times \NN| \leq |S|$. So $|\CC| = |S|$. (If it's finite, we
            can see that $|\overline{\QQ(S)}| \leq |\overline{\QQ(\NN)}| = |\NN|
            < |\CC|$).


         \item Any permutation of a transcendence basis can be extended to an
            automorphism over $\QQ$, and clearly the permutation group of an
            uncountable set is uncountable.
      
      \end{enumerate}

\end{enumerate}
\end{document}
