\documentclass{article}
\usepackage{amsmath,amsfonts,amssymb}
\usepackage{fullpage}
\usepackage{mathrsfs}

\newcommand\ZZ{\mathbb Z}
\newcommand\RR{\mathbb R}
\newcommand\CC{\mathbb C}
\renewcommand\S{\mathcal S}
\DeclareMathOperator\Hom{Hom}

\title{Algebra Homework 1}
\author{Mendel Feygelson}

\begin{document}
\maketitle
\begin{enumerate}

  \item Let $m_1,\dotsc,m_r$ be the generators. We know that $n$-fold tensor
     products of the $m_i$ generate $T^n(M)$ and that $T^n(M)$ surjects onto
     $\bigwedge^n(M)$ by $x_1\otimes\dotsb\otimes x_n \mapsto
     x_1\wedge\dotsb\wedge x_n$, so $\bigwedge^n(M)$ is generated by $n$-fold
     wedge products of the $m_i$. But an $n$-fold wedge product whose terms are
     drawn from only $r$ distinct elements must have a repeated term, and thus
     must be zero.

  \item Denote the right hand side by $\S$. First we define a symmetric
     $n$-linear map $(M \oplus N)^n \to \S$ by $(m_1 \oplus n_1,\dotsc,m_n
     \oplus n_n) \mapsto \sum_{I \subset \{1,\dotsc,n\}} \prod_{i \in I} m_i
     \otimes \prod_{j \not\in I} n_j$ (if you'll excuse my use of $n$ as both an
     index and an element of $N$). This gives us a map $S^n(M \otimes N) \to
     \S$. 
     %It's surjective since $(m_1 \oplus 0,\dotsc,m_i \oplus 0, 0 \oplus
     %n_{i+1},\dotsc,0 \oplus n_n) \mapsto m_1 \dotsm m_i \otimes n_{i+1} \dotsm
     %n_n$ and these elements generate. And we can see that it's injective
     %because 
     %The inverse map is given on each component by sending $m_1\dotsm m_m
     %\otimes n_{m+1}\dotsm n_n \mapsto (m_1 \oplus 0)\dotsm(m_m \oplus 0)
     %(0 \oplus n_{m+1}) \dotsm (0 \oplus n_n)$. (The corresponding map from
     %$S^m(M) \times S^{n-m}(N)$ is bilinear and so this map is well defined.)
     %It's easy to see that these maps are mutually inverse.
     
     To get the inverse map, for fixed $m_1,\dotsc,m_m \in M$, we define a map
     $N^{n-m} \to S^n(M \oplus N)$ taking $n_{m+1},\dotsc,n_n \mapsto (m_1
     \oplus 0)\dotsm(m_m \oplus 0)(0 \oplus n_{m+1})\dotsm(0 \oplus n_n)$. This
     is symmetric multilinear in the $n_j$ so we get a map $S^{n-m}(N) \to S^n(M
     \oplus N)$. Thus we have a map $M^m \to \Hom(S^{n-m}(N), S^n(M \oplus N))$.
     This map is symmetric multilinear, so yields a map $S^m(M) \to
     \Hom(S^{n-m}(N), S^n(M \oplus N))$. And since this is a linear map, it
     corresponds to a map $S^m(M) \otimes S^{n-m}(N) \to S^n(M \oplus N)$.
     Applying such a map in each component gives us the inverse map.

     It's not so hard to see that these maps are inverses since they send $(m_1
     \oplus 0)\dotsm(m_m \oplus 0)(0 \oplus n_{m+1})\dotsm(0 \oplus n_n) \mapsto
     m_1\dotsm m_m \otimes n_{m+1} \dotsm n_n$ and back, and these elements
     generate.

  \item $\bigwedge^{i+j}(M) = \bigwedge^i(M) \wedge \bigwedge^j(M)$, so we have
     a quotient map $\bigwedge^i(M) \otimes \bigwedge^j(M) \to
     \bigwedge^{i+j}(M)$. So we get the desired map because $\otimes$ is left
     adjoint to $\Hom$. And it's clearly unique from the specification.

  \item Fix a basis $x_1,\dotsc,x_r$ for $M$. Then $x_1 \wedge \dotsb \wedge
     x_r$ is a basis for $\bigwedge^r(M)$, so we can identify $\bigwedge^r(M)$
     with $M$ for the remainder of this problem, and so we can identify
     $\Hom(\bigwedge^j(M),\bigwedge^r(M))$ with $(\bigwedge^j(M))^*$.

     Let $\eta = x_{n_1} \wedge \dotsb \wedge x_{n_j} \in \bigwedge^j(M)$. Such
     elements span $\bigwedge^j(M)$. Let $\omega = \pm x_{m_1} \bigwedge \dotsb
     \bigwedge x_{m_i} \in \bigwedge^i(M)$ consist of the remaining basis
     elements, such that $\omega \wedge \eta = 1$. Note that for any other such
     element $\nu$ of $\bigwedge^j(M)$ which is made up of a different set of
     basis elements than $\eta$, and so isn't a multiple of $\eta$,  $\omega
     \wedge \nu = 0$. Thus we see that the image of $\omega$ is $\eta^*$. So the
     image of $\bigwedge^i(M)$ spans $(\bigwedge^j(M))^*$. And because
     $\binom{r}{i} = \binom{r}{j}$, $\dim\bigwedge^i(M) = \dim\bigwedge^j(M)$,
     and so they're isomorphic.

  \item We define the desired map from $M^n$. It's not hard to see that it's
     linear -- for any given $i$, in each summand, $x_i$ appears once either as
     $f(x_i)$ or as a component of the wedge product, and in either case the
     summand is linear in $x_i$. And if $x_j = x_k = x$, then when $i \neq j,k$,
     the wedge product in the summand is 0, so the sum reduces to
     $f(x)((-1)^{j-1}x_1\wedge\dotsb\hat x_j\wedge\dotsb\wedge x_n +
     (-1)^{k-1}x_1\wedge\dotsb\hat x_k\wedge\dotsb\wedge x_n) = 0$ by
     cancellation. So we get $\phi_n(f)$ as desired.

     And when you take $\phi_{n-1}(f) \circ \phi_n(f)$, then for each pair of
     indices $i<j$, you get one summand from when $x_i$ is removed first, and
     one from when $x_j$ is removed first. In the former case the sign
     coefficient is $(-1)^{i-1}(-1)^{j-2}$ and in the latter it's
     $(-1)^{i-1}(-1)^{j-1}$, but otherwise the two terms are the same, so they
     cancel.

  \item Let $x_1,\dotsc,x_r$ be a basis for $M$. Suppose $z$ is in the center of
     $T(M)$. Write $z = \sum a_Ix_I$. $\sum a_I x_1\otimes x_I = x_1\otimes z =
     z \otimes x_1 = \sum a_I x_I \otimes x_1$. So each $x_I$ of degree $>0$,
     whose coefficient $a_I$ is nonzero, looks like $x_1 \otimes \dotsb$. But
     an identical argument shows that each such $x_I$ also looks like $x_2
     \otimes \dotsb$. But such tensors cannot be equal, so there can be no such
     terms in the sum, so $z \in R$.

  \item $\ZZ$ is a PID so prime ideals are maximal. Thus anything in the
     Jacobson radical of $\ZZ$ would have to be divisible by every prime. The
     only such number is 0.

  \item $M_2(k)$ is simple but $\begin{pmatrix}0&1\\0&0\end{pmatrix}$ is
     nilpotent.

\end{enumerate}
\end{document}
