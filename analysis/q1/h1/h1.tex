\documentclass{article}
\usepackage{amsmath,amsfonts,amssymb}
\usepackage{fullpage}
\usepackage{mathrsfs}

\newcommand\ZZ{\mathbb Z}
\newcommand\RR{\mathbb R}
\newcommand\CC{\mathbb C}
\newcommand\C{\mathscr C}
\renewcommand\P{\mathscr P}
\newcommand\F{\mathscr F}
\newcommand\NN{\mathbb N}
\newcommand\G{\mathscr G}
\newcommand\J{\mathscr J}
\newcommand\B{\mathscr B}
\newcommand\A{\mathscr A}

\title{Analysis Homework 1}
\author{Mendel Feygelson}

\begin{document}
\maketitle
\begin{enumerate}

  \item 

    \begin{enumerate}

      \item $2^{|X|}$
      \item $2^N$ -- we can assume that the elements of $\C$ are disjoint by replacing $C_i$ with $C_i \setminus \bigcup_{j=1}^{i-1} C_j$ -- this is a collection of $N$ disjoint subsets that generates the same algebra. And we can then think of each of the $C_i$ as singletons, since the algebra operations can't separate their elements. And then the algebra generated by $\C$ is just $\P(\bigcup\C)$.
      \item This follows from a similar argument to part (b). The only part that could fail is that one of the disjoint $C_i$ could be empty, but if we throw away all the empty ones, we still get $2^n$ possible subsets, where $n$ is the number of nonempty $C_i$.
      \item Probably not -- if $\F$ is a $\sigma$-algebra generated by countably many elements, then we could use the same procedure to make them disjoint. But then the $\sigma$-algebra operations can't separate their elements so we can once more treat them as singletons. But then we just get $\P(\NN)$, which is uncountable.

    \end{enumerate}
    
  \item It's an elementary fact that inverses preserve all of the algebra operations: $f^{-1}(A^c) = f^{-1}(A)^c$, $f^{-1}(\bigcup A_i) = \bigcup f^{-1}(A_i)$, and that's really all we need.
  
  \item We need to check that $\F$ is a $\sigma$-algebra. Clearly $\F$ is closed under complements. And countable unions of countable sets are countable, and any union one of whose terms has countable complement, has countable complement. So $\F \supset \sigma(\P)$. And clearly $\sigma(\P)$ contains all countable sets, and therefore all sets with countable complements, so $\sigma(\P) \supset \F$.
  
  \item The union of two overlapping such intervals is just another such interval, so we can write elements of $\J$ as $\bigcup_{i=1}^n (a_i,a_i']$ with $a_i<a_i'<a_j$ for $i<j$ (if $a_i'=a_j$, we can also combine to the interval $(a_i,a_j']$). And the complement of such an element is just $(-\infty,a_0] \cup (a_0',a_1] \cup \cdots \cup (a_{n-1}',a_n] \cup (a_n',\infty)$, is an element of $\J$. So $\J$ is an algebra (it's clearly closed under finite unions since finite unions of finite unions are finite unions), and so clearly $\J$ is the smallest $\sigma$-algebra containing $\J_0$.
  
  $(a,b) = \bigcup_{n \in \NN} (a,b-\frac1n] \in \J$, so $\sigma(\J) \supset \B(\RR)$, and similarly $(a,b] = \bigcap (a,b+\frac1n] \in \B(\RR)$, so $\B(\RR) \supset \sigma(\J)$.
  
  \item Let $B_n = \bigcap_{k=n}^\infty A_k$. Note that $B_i \subset B_{i+1}$. Let $B = \bigcup B_i$. $\mu(B) = \sum_{j=0}^\infty \mu(B_j \setminus B_{j-1}$, while $\mu(B_i) = \sum_{j=0}^i \mu(B_j \setminus B_{j-1})$, so $\mu(B_i) \to \mu(B)$. So $\mu(\liminf A_n) = \mu(B) = \lim \mu(B_i)$. But $\mu(B_i) \leq \inf{k \geq i} \mu(A_i)$, since $B_i \subset A_k$ for every $k \geq i$, so we get that $\mu(\liminf A_n) \leq \liminf \mu(A_n)$.
  
  If $\mu(X) < \infty$, then since  $\limsup A_n = (\liminf A_n^c)^c$, $\mu(\limsup A_n) = \mu(X) - \mu(\liminf A_n^c) \geq \mu(X) - \liminf (\mu(X) - \mu(A_n)) = \limsup \mu(A_n)$.
  
  However, if $\mu(X) = \infty$, we can take $A_n = (n,\infty)$. Then $\limsup A_n = \emptyset$, but $\mu(A_n) = \infty$.
  
  \item Suppose $A \in \F$ has positive measure. Since $A$ is not an atom, we can find $B \subset A$ such that $0 < \mu(B) < \mu(A)$. Now, $\mu(B) + \mu(A \setminus B) = \mu(A)$, so one of them has measure $\leq \frac12\mu(A)$. Repeating this procedure, we can get sets of arbitrarily small positive measure.
  
  \item We can translate intervals, so if $\{I_i\}$ is a cover of $A$, then $\{I_i+x\}$ is a cover of $A+x$ of the same total length. So $m^*(A+x) \leq m^*(A)$ and vice versa.
  
  If we can cover $B$ by collections of intervals of arbitrarily small total length, then if we take of cover of $A$ by intervals of total length arbitrarily close to $m^*(A)$, and add in a cover of $B$ by intervals of arbitrarily small total length, then we have a cover of $A \cup B$ by intervals of total length arbitrarily close to $m^*(A)$. So $m^*(A \cup B) = m^*(A)$.
  
  Finally, suppose $A$ is Lebesgue measurable, and $B$ is an arbitrary subset. Then $m^*(B \cap (A+x)) + m^*(B \setminus (A+x)) = m^*((B-x) \cap A) + m^*((B-x) \setminus A) = m^*(B-x) = m^(B)$, so $A+x$ is measurable, and of course $m(A+x) = m(A)$.
  
  \item In the first case the functions that are constant on each $C_i$ are measurable -- clearly such functions are measurable, since then $f^{-1}$ of any set is just a union of $C_i$, and conversely, supposing $f(C_i)$ has two points, $a < b$, then $f^{-1}(-\infty,\frac{a+b}2)$ has some points of $C_i$ and not others, and such a set does not arise in $\sigma(\C)$ (it's easy to see that $\sigma(\C)$ is just the collection of finite unions of $C_i$). In the second case, every function is measurable.
  
  \item This is just $g^{-1}(0)$, where $g$ is the function $\limsup f_n - \liminf f_n$ on the set $(\limsup f_n)^{-1}(\RR) \cap (\liminf f_n)^{-1}(\RR)$.
  
  \item For the first part, just take the union of a cover of $A$ by open intervals of total length less than $\mu(A) + \epsilon$.
  
  For the second part, we split $\RR$ into countably many closed intervals $I_n$ of length 1 that intersect on a set of measure 0. For each $I_n$, we have an open set $O_n$ containing $I_n \setminus A$ such that $\mu(O_n) \leq 1 - \mu(A \cap I_n) + \frac\epsilon{2^n}$, so if $F_n = I_n \setminus O_n$, then $F_n \subset A \cap I_n$, $F_n$ is closed, and $\mu(F_n) \geq \mu(A \cap I_n) - \frac\epsilon{2^n}$. Then $\sum \mu(F_i) \geq \mu(A) - \epsilon$, and so some $\sum_{i=1}^n \mu(F_i) \geq \mu(A) - 2\epsilon$, so we can take $F = \bigcup_{i=1}^n F_i$ if we want compactness.
  
  Then we can take $B_i$ to be the union of all $F \subset A$ closed, and $B_e$ to be the intersection of all $O \supset A$ open.

\end{enumerate}
\end{document}
