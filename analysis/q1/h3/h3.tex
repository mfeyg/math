\documentclass{article}
\usepackage{amsmath,amsfonts,amssymb}
\usepackage{fullpage}
%\usepackage{mathrsfs}

\newcommand\ZZ{\mathbb Z}
\newcommand\RR{\mathbb R}
\newcommand\CC{\mathbb C}
\newcommand\NN{\mathbb N}

\DeclareMathOperator\im{im}

\title{Analysis Homework 3}
\author{Mendel Feygelson}

\begin{document}
\maketitle
\begin{enumerate}

  \item We know from class that for any $\varepsilon$ there exists a $g : \RR
     \to \RR$ continuous with compact support such that $\int_\RR |f-g| <
     \varepsilon$. Let $\lambda$ be the measure of a compact set which supports
     $g$. $g$ is uniformly continous, so there is a $\delta$ such that
     $|g(x+h)-g(x)|<\varepsilon$ for $h < \delta$, so $\int_\RR
     |g(x+h)-g(x)|\,dx < 2\lambda\varepsilon$. So if $h<\delta$, then $\int_\RR
     |f(x+h)-f(x)|\,dx \leq \int_\RR |f(x+h)-g(x+h)| + \int_\RR |g(x+h)-g(x)| +
     \int_RR |g(x)-f(x)| < 2(\lambda+1)\varepsilon \to 0$ as $\varepsilon \to
     0$.

  \item First we want to show that $\phi$ is continuous. We have that
     $\phi(\lambda(x-y)+y) \leq \lambda(\phi(x)-\phi(y)) + \phi(y)$, so
     $\frac{\phi(\lambda(x-y)+y)-\phi(y)}\lambda \leq \phi(x)-\phi(y)$.
     Substituting $\lambda=\frac{z-y}{x-y}$, we get that
     $\frac{\phi(z)-\phi(y)}{z-y} \leq \frac{\phi(x)-\phi(y)}{x-y}$ for
     $y < z \leq x$ and $\frac{\phi(z)-\phi(y)}{z-y} \geq
     \frac{\phi(x)-\phi(y)}{x-y}$ for $y > z \geq x$. So,
     $\frac{\phi(x)-\phi(y)}{x-y}$ is increasing (nondecreasing) in $x$ for any
     fixed $y \neq x$, and by symmetry, it's also increasing in $y$ for fixed
     $x$. So if $x'<x<y<y'$ then $\frac{\phi(x')-\phi(y)}{x'-y} \leq
     \frac{\phi(x)-\phi(y)}{x-y} \leq \frac{\phi(x)-\phi(y')}{x-y'}$, so
     $\frac{\phi(x)-\phi(y)}{x-y}$ is bounded for $x$ and $y$ in this range, so
     $\phi(x)-\phi(y) \to 0$ as $x-y \to 0$.

     Next, suppose $s : X \to \RR$ is a simple function. We claim that
     $\phi(\int s) \leq \int \phi \circ s$. We proceed by induction on the
     number of terms of $s$ (that is, $|\im(s)|$). In the base case, $s$ is
     constant, say $s \equiv a$, then $\phi(\int_X s) = \phi(a) = \int_X \phi(a)
     = \int_X \phi \circ s$. Suppose $s$ has more than one term. We write $X$ as
     the disjoint union of two measurable sets $A \cup B$, where $s|_A$ and
     $s|_B$ are each simple functions with fewer terms than $s$ (that is, fewer
     values in their images). Then, $\phi(\int_X s) = \phi(\int_A s + \int_B s)
     = \phi\left(\mu(A) \int_A s \, \frac{d\mu}{\mu(A)} + \mu(B) \int_B s \,
     \frac{d\mu}{\mu(B)}\right) \leq \mu(A)\phi\left(\int_A s\,
     \frac{d\mu}{\mu(A)}\right) + \mu(B)\phi\left(\int_B s\,
     \frac{d\mu}{\mu(B)}\right) \leq \mu(A)\int_A (\phi \circ s) \,
     \frac{d\mu}{\mu(A)} + \mu(B)\int_B (\phi \circ s) \, \frac{d\mu}{\mu(B)} =
     \int_X \phi \circ s$, where the second inequality follows by induction
     because $(A,\frac\mu{\mu(A)})$ and $(B,\frac\mu{\mu(B)})$ are each measure
     spaces of unit measure on which $s$ is a simple function that has fewer
     terms.

     Now this is where this proof gets too messy. For $k \in \NN$, we define a
     simple function $s_k$ as follows: For $n,m \in \ZZ$, $|n|,|m| < k$, let $A
     = \phi^{-1}[n,n+1] \cap [m,m+\frac1k]$. If $A \neq \varnothing$, we pick $y
     \in A$ where $\phi$ achieves its maximal value, and define $s_k$ to be $y$
     on $f^{-1}(A)$. If $x$ is not in any such $A$, we set $s_k(x)=0$.  Then for
     any $x \in X$, for $k$ sufficiently large, $x$ will be in some $A$, so
     $|f(x) - s_k(x)| \leq \frac1k$, so $s_k \to f$, and so $\phi \circ s_k \to
     \phi \circ f$. And we also have that $|s_k| \leq |f|+1$, which is
     integrable on $X$ since $X$ has finite measure; $|\phi \circ s_k| \leq
     \max\{|\phi(0)|,|\phi \circ f| + 1\}$; and $\phi \circ s_k \geq \phi \circ
     s_{k+1} \geq \phi \circ f$ for every $k$.

     Now we compute, $\phi(\int_X f) = \phi(\int_X \lim s_k) = \lim \phi(\int_X
     s_k) \leq \lim \int_X \phi \circ s_k$. Now if $\int_X \phi \circ f =
     \infty$, then this value is vacuously $\leq \int_X \phi \circ f = \infty$.
     If it's finite, we have that $\max\{|\phi(0)|,|\phi \circ f| + 1\}$ is
     integrable on $X$, so this value is $\int_X \lim \phi \circ s_k = \int_X
     \phi \circ f$. And if it's $-\infty$, then on $\{\phi \circ f \geq 0\}$,
     $\phi \circ s_k$ is dominated by an integrable function, and on $\{\phi
     \circ f < 0\}$, $\phi \circ s_k$ is eventually a monotonically decreasing
     sequence of negative functions (at least on a set where $\phi \circ f <
     \varepsilon$ and $\phi \circ s_k$ converges uniformly, but such a set
     should be arbitrarily large), so again we should get that $\lim \int_X \phi
     \circ s_k = \int_X \phi \circ f$.

  \item This space intentionally left blank.

  \item Note that by the triangle inequality, we see that refining a partition
     will increase the total variation on that partition (and so provide a
     better estimate of the total variation). We set $v_\varepsilon =
     V(f)_a^{x_0-\varepsilon} + |f(x_0) - f(x_0-\varepsilon)|$, that is the
     supremum of the variation over all partitions of $[a,x_0]$ whose last two
     terms are $x_0-\varepsilon < x_0$. Any partition whose last terms are
     $a_{n-1} < a_n = x_0$, with $a_n - a_{n-1} > \varepsilon$ can be refined to
     a partition with last terms $x_0-\varepsilon < x_0$, so we thus see that
     $\lim_{\varepsilon\to0}v_\varepsilon = V(f)_a^{x_0}$. But since $f$ is
     continuous at $x_0$, we have that $|f(x_0) - f(x_0-\varepsilon)| \to 0$ as
     $\varepsilon \to 0$, so we get that $V(f)_a^{x} \to V(f)_a^{x_0}$ as $x \to
     x_0$ from below.
     
     Note that for any $x<y<z$, $V(f)_x^y + V(f)_y^z \leq
     V(f)_x^z$ -- we can see this from the definition of $V$. Thus,
     $V(f)_x^{x_0} \to 0$ as $x\to x_0$ from below. And by symmetry (consider
     $g(x)=f(-x)$), we also have that $V(f)_{x_0}^x \to 0$ as $x_0 \to x$ from
     above. But then $V(f)_a^x \leq V(f)_a^{x_0} + V(f)_{x_0}^x \to
     V(f)_a^{x_0}$, so $V(f)_a^x \to V(f)_a^{x_0}$ as $x \to x_0$ from above as
     well.
     
  \item Consider the decreasing sequence of values $x_n = \frac2{(2n+1)\pi}$.
     $|f(x_n)-f(x_{n+1})| = |x_n\sin( (2n+1)\frac\pi2) - x_{n+1}\sin( (2n+3)
     \frac\pi2)| = \frac1\pi(\frac1{2n+1} + \frac1{2n+3})$. And the sum of such
     terms diverges as we add more of these values to our partition.

  \item Let $I$ be the indicator function of an interval $[a,b]$. Then $\int
     I(x)e^{ixt}\,dx = \int_a^b e^{ixt}\,dx = frac1{it}(e^{ibt} - e^{iat}) \to
     0$ as $t \to \infty$. Now we write $f = g + \tilde f$ where $g$ has compact
     support and $\int \tilde f < \varepsilon$. Then $|g|$ can be written as the
     sum of indicator functions (of subsets of a compact interval) $g = \sum
     I_n$. So $\int g(x)e^{ixt}\,dx = \sum \int I_n(x)e^{ixt}\,dx$. This sum is
     finite and each of its terms gets arbitrarily small as $t \to \infty$, so
     $\int g(x)e^{itx}\,dx \to 0$ as $t \to \infty$, so $\int f(x)e^{itx}\,dx
     \to 0$ as $t \to \infty$.

  \item On the one hand, for any partition $a=x_0<\dotsb<x_n=b$, the total
     variation over that partition is $\sum |F(x_i)-F(x_{i-1})| = \sum
     |\int_{x_{i-1}}^{x_i} f| \leq \sum \int_{x_{i-1}}^{x_i} |f| = \int_a^b
     |f|$. Now, we want to find partitions on which the total variation gets
     arbitrarily close to $\int_a^b |f|$. Write $f = g + \tilde f$, where
     $g$ is continuous and $\int |g-\tilde f| < \varepsilon$. $\{g \neq 0\}$ is
     the countable union of open intervals on each of which $g$ is either
     non-positive or non-negative.  Choosing the endpoints of these intervals as
     the points of our partition, we get  $\sum |\int_{x_{i-1}}^{x_i} g| = \sum
     \int_{x_{i-1}}^{x_i} |g| = \int_a^b |g|$. So if we look at the supremum as
     we put all the endpoints into our partition, we get total variation $\sum
     |\int_{x_{i-1}}^{x_i} f| = \sum |\int_{x_{i-1}}^{x_i} g+\tilde f| \geq \sum
     |\int_{x_{i-1}^{x_i}} g| - \varepsilon = \int_a^b |g| - \varepsilon \geq
     \int_a^b |f| - 2\varepsilon$.

  \item Firstly, $f$ is integrable because $\int f \cdot 1 \leq C$. Let $\tilde
     g$ be 1 on $\{f > 0\}$ and $-1$ on $\{f \leq 0\}$. Let $\{f > 0\} = U \cup
     N$ where $U$ is open and $m(N) < \varepsilon$. We can define a continuous
     function $g$ which is 1 on $U$ and $-1$ on all of $U^c$ except for a set
     $M$ with $m(M) < \varepsilon$ (and $|g| \leq 1$ everywhere). Then away from
     $M \cup N$, we have that $\int |f| = \int fg \leq C$. And we can choose $M$
     and $N$ small enough that $\int |f|$ on them is arbitrarily small.

  \item By simple algebra, we see that for rational $x$, $f(x) = Cx$, where $C =
     f(1)$ (we have $f(\frac mn) = mf(\frac1n)$ and $nf(\frac1n)=f(1)$). Now,
     $\int_a^b f = \sup_{(q,r) \subset (a,b)} \int_q^r f = C(a-b)$ (where $q,r$
     are rational). So $f(x) = Cx$ almost everywhere. Let $N = \{f(x) \neq
     Cx\}$. Note that if $x,y \in N^c$, then $x+y \in N^c$, so if $x \in N^c$,
     then $N - x \subset N$. So if $\alpha \in N$, then $\alpha - N^c \subset
     N$, but $m(\alpha-N^c) = \infty$ -- a contradiction. So $N = \varnothing$.
     
  \item $d(f_n,f) = \int_X \frac{|f_n-f|}{1+|f_n-f|}$. If $d(f_n,f) \to 0$ then
     clearly $\mu\{|f_n-f|>\varepsilon\} \to 0$. And conversely, if
     $\mu\{|f_n-f|>\varepsilon\} \to 0$ then the integral over
     $\{|f_n-f|\leq\varepsilon\}$ is bounded by
     $\mu(X)\frac\varepsilon{1-\varepsilon} \to 0$ as $\varepsilon \to 0$ and
     the integral over $\{|f_n-f|>\varepsilon\}$ is bounded by
     $\mu\{|f_n-f|>\varepsilon\} \to 0$ as $n \to \infty$.

\end{enumerate}
\end{document}
